\chapter*{Введение}
Графический движок (англ. graphics engine) — программное обеспечение,
основной задачей которого является визуализация (рендеринг) двухмерной или трёхмерной
компьютерной графики. Графические движки, использующееся в программах по работе с компьютерной графикой,
обычно называются «рендерерами», «отрисовщиками» или «визуализаторами».
Само название «графический движок» используется, как правило, в компьютерных играх.

Основное отличие игровых графических движков от неигровых состоит в том, что первые должны работать в режиме реального времени,
тогда как вторые могут тратить много времени
на вывод одного изображения. Также существенным отличием является то, что графические движки производят визуализацию с помощью
графических процессоров (GPU), неигровые используют только центральные процессоры (CPU).

Основная цель данной работы заключается в разработке библиотеки, содержащей инструменты, необходимые для
создания игрового графического движка. Актуальной работу делает гибкость и одновременная простота использования инструментов.

Некоторые возможности и особенности библиотеки:
\begin{enumerate}
\item Поддержка шейдеров.
\item Отрисовка сцены.
\item Поддержка моделей.
\item Контроль и обход большинства ошибок.
\item Кроссплатформенность.
\end{enumerate}
Также целью работы является приобретение автором некоторых знаний спецификации OpenGL.

Задачи реализуются на языке D - объектно-ориентированном, мультипарадигменном языке программирования, 
создателем которого является Уолтер Брайт. 